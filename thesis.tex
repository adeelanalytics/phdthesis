\documentclass[11pt]{book}

\usepackage[T1]{fontenc}
\usepackage{hyperref}
\usepackage{microtype}
\usepackage{lettrine}

\usepackage[commands,environments,enumerate,citations,notes,a4paper]{AVT}

\bibliography{citations}

\title{Gaussian Processes and Statistical Decision-making in Non-Euclidean Spaces}
\author{Alexander Terenin}
\date{August 2021}

\begin{document}

\begin{titlepage}
\maketitlehooka
\centering
\huge
\null
\vfill
\thetitle
\par
\vfill
\LARGE
\theauthor
\par
\large
Department of Mathematics
\par
Imperial College London
\par
\vfill
\null
\vfill
a dissertation submitted for the degree of
\par
Doctor of Philosophy
\par
\strut
\par
\thedate
\par
\vfill
\null
\maketitlehookd
\end{titlepage}

\chapter*{Declaration}

No more than 100,000 words

\chapter*{Copyright}

The copyright of this thesis rests with the author. Unless otherwise indicated, its contents are licensed under a Creative Commons Attribution 4.0 International Licence (CC BY).

Under this licence, you may copy and redistribute the material in any medium or format for both commercial and non-commercial purposes. You may also create and distribute modified versions of the work. This on the condition that you credit the author.

When reusing or sharing this work, ensure you make the licence terms clear to others by naming the licence and linking to the licence text. Where a work has been adapted, you should indicate that the work has been changed and describe those changes.

Please seek permission from the copyright holder for uses of this work that are not included in this licence or permitted under UK Copyright Law.

\chapter*{Acknowledgments}

\chapter*{Abstract}

Not more than 300 words

\tableofcontents





\chapter{Introduction}

\lettrine{M}{achine learning} provides a variety of frameworks for reasoning about unknown quantities of interest on basis of data, but most of these focus on pure learning, with comparatively less focus on decision

Gaussian processes are some of the simplest models usable for decision, but working with them takes effort, particularly if one is in a non-vanilla setting

This thesis: make Gaussian processes easier to work with through better numerics, and expand the settings they are usable in

\section{Gaussian processes}

Here we describe the GP formalism from scratch, starting from the simplest and ending with the most general and abstract setting

\subsection{Gaussian random variables}

Gaussian density

\subsection{Gaussian random vectors}

Matrix square root

\subsection{Gaussian random functions}

Let $X$ be a set. 

\subsection{Gaussian processes in general vector spaces}

Duality

Covariance form

Covariance operator

\section{Bayesian learning}

Posterior

Variational inference

\section{Statistical decision-making}

Decision

\subsection{Multi-armed bandits}

Setup

Basic regret analysis

\subsection{Bayesian optimization}

Setup

Regret analysis

\section{Contributions}

This work is either published or under review: \textcite{wilson20,borovitskiy20,borovitskiy21,wilson21,hutchinson21}.



\chapter{Pathwise Conditioning}

\section{Introduction}

The standard view of GP conditioning is to think about distributions

In the early 1970s, an alternative view emerged: work with RVs directly

This view, for whatever reason, never made it into the ML literature

This chapter of the thesis will explore its consequences and use it to solve what has been an open issue in Bayesian optimization for quite some time

The ideas in this chapter are published at ICML and JMLR

\section{Conditioning multivariate Gaussians}

Defn of MVN

\subsection{Distributional conditioning}

Formulas

Discussion

\subsection{Pathwise conditioning}

Formulas

Proof

Discussion

Second proof

Discussion

\section{Conditioning Gaussian processes}

Brief recollection of GPs 

\subsection{Distributional conditioning}

Formula

Discussion

\subsection{Pathwise conditioning}

Formula

Discussion

\section{Sampling from prior Gaussian processes}

The previous formula suggests an interesting class of approximations: discretize the prior

Finite basis expansion

This section: explore different expansions

\subsection{Random feature methods}

Construct a finite-dimensional feature map by discretizing the white noise integral

Closely related to stationarity in embedded spaces

\subsection{Karhunen--Loéve expansions}

Alternatively: work with a different kind of spectral theory directly in the space

Obtain the KL expansion

\subsection{Finite element methods}

Completely different idea: represent GP as SPDE and solve the SPDE

Derivation

\section{Approximating pathwise data-dependent terms}

Previous part as about approximating the prior, which gets us to linear test-time costs

What about reinterpreting training-time costs from this viewpoint?

\subsection{Inducing points}

Reinterpret classical constructions

\subsection{Approximations which change the model}

RFF GP can be reinterpreted as replacing the prior with a finite-dim version

This results in unnecessarily large approximation error and variance starvation

\section{Error analysis}

Theorems

\section{Applications}

\subsection{Bayesian optimization}

Experiment

\section{Discussion}

Very useful technique for optimizing GP trajectories

Lets you do lots of awesome stuff

This work lets us understand it a lot better





\chapter{Non-Euclidean Matérn Gaussian Processes}

\section{Riemannian Matérn Gaussian Processes}

\subsection{Review of Riemannian geometry}
\subsection{The Laplace--Beltrami operator}
\subsection{A no-go theorem for kernels on manifolds}
\subsection{Stochastic partial differential equations}
\subsection{The Riemannian Matérn kernel}
\subsection{Illustrated examples}

\section{Graph Matérn Gaussian Processes}

\subsection{Review of graph theory}
\subsection{The graph Laplacian}
\subsection{The graph Matérn kernel}
\subsection{Illustrated examples}

\section{Gaussian Vector Fields on Riemannian Manifolds}

\subsection{Review of vector fields on manifolds}
\subsection{Gauge-equivariant projected kernels}
\subsection{Illustrated examples}

\section{Discussion}





\chapter{Conclusion}

\printbibliography

\end{document}
